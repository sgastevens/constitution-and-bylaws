\documentclass[12pt]{scrreprt}
\usepackage{tocloft}
\usepackage{xcolor}
\usepackage{hyperref}
\usepackage{lineno}
\usepackage{hyperref}

\setcounter{tocdepth}{1} %Adjust to change what level headings are shown in table of contents
\setcounter{secnumdepth}{5}

%Setup display of section headings in article
\renewcommand\thechapter{Article \Roman{chapter}}
\renewcommand\thesection{Section \arabic{chapter}.\arabic{section}}
\renewcommand\thesubsection{\Alph{subsection}}
\renewcommand\thesubsubsection{\alph{subsubsection}}
\renewcommand\theparagraph{\arabic{paragraph}}
\renewcommand\thesubparagraph{\alph{subparagraph}}

%Setup display of section headings in referrences and table-of-contents
\makeatletter
\renewcommand{\p@chapter}{Article \Roman{chapter}\expandafter\@gobble}
\renewcommand{\p@section}{Section \arabic{chapter}.\arabic{section}\expandafter\@gobble}
\renewcommand{\p@subsection}{Section \arabic{chapter}.\arabic{section}(\Alph{subsection})\expandafter\@gobble}
\renewcommand{\p@subsubsection}{Section \arabic{chapter}.\arabic{section}(\Alph{subsection})-\alph{subsubsection}\expandafter\@gobble}
\renewcommand{\p@paragraph}{Section \arabic{chapter}.\arabic{section}(\Alph{subsection})-\alph{subsubsection}.\arabic{paragraph}\expandafter\@gobble}
\renewcommand{\p@subparagraph}{Section \arabic{chapter}.\arabic{section}(\Alph{subsection})-\alph{subsubsection}.\arabic{paragraph}.\alph{subparagraph}\expandafter\@gobble}
\makeatother

%Setup padding for headings in table-of-contents
\setlength{\cftchapnumwidth}{6em}
\setlength{\cftsecnumwidth}{6em}
\setlength{\cftsubsecnumwidth}{1.8em}
\setlength{\cftsubsubsecnumwidth}{1.8em}
\setlength{\cftparanumwidth}{1.8em}
\setlength{\cftsubparanumwidth}{1.8em}

%Setup coloring for referrences and table-of-contents
\hypersetup{
  colorlinks=true,
  citecolor=blue,
  filecolor=blue,
  linkcolor=blue,
  urlcolor=blue, 
  pdfborder={0 0 0}
}

\begin{document}
\title{The Bylaws \\ of the \\ Student Government Association \\ of \\ Stevens Institute of Technology}
\subtitle{}
\author{}
\date{}
\maketitle

%auto-generate table-of-contents
\tableofcontents

%start displaying line numbers
\modulolinenumbers[1]
\linenumbers

\chapter{Formation of Law}

\section{Legislation}

\subsection{Definition}
Legislation is a written proposal brought before the Senate by a Senator to take certain action or adopt, remove, or amend certain policies.

\subsection{Types}
Senators can introduce only two types of legislation to the Senate: resolutions and proclamations. A resolution can adjust fund allocations, modify governing documents of the SGA, or it can mandate action of SGA Officers. A proclamation is a decree, declaring an opinion of the Senate; it may address a specific party. Any piece of legislation that is under the consideration of the Senate shall be referred to as a bill. Any passed piece of legislation shall be referred to by its respective type: resolution or proclamation.

\section{Origin of a Bill}
A senator shall submit a bill to the Secretary, who shall place the bill on the docket. A motion to approve the bill must appear on the docket for the Senate to consider the bill. The Senate can approve only bills on the docket.

\section{Structure} \label{sec:bill_content}
\subsection{General}
A piece of legislation shall contain:
\begin{itemize}
    \item a title;
    \item the proposal date
    \item the proposer(s);
    \item the sponsor(s), if any;
    \item the reasons for the legislation, in the form of a “whereas” clauses; and
    \item the intended actions, in the form of “resolved” clauses.
\end{itemize}
    

\subsection{Title}
The proposer(s) of legislation shall title their legislation in accordance to the following
structure:
\begin{enumerate}
    \item The type of legislation, either “Proclamation” or “Resolution”;
    \item An index code with a hyphen (“-”) separating each of the following subitems:
    \begin{enumerate}
        \item “S” for Resolution and “P” for Proclamation;
        \item The last two digits of the current year — i.e. “17” for 2017 — followed by the abbreviation for the semester in which the proposer(s) presents the legislation — i.e. “S” for Spring or “F” for Fall;
        \item A three-digit ordinal number, incrementing for each bill presented to the Secretary; this number shall be prefixed with zeros to the meet three-digit requirement when necessary — i.e. “005” for the fifth bill introduced into the Senate;
    \end{enumerate}
    \item The phrase “of the”, followed by the current Senate session number, followed by “Senate of the Student Government Association of Stevens Institute of Technology”;
    \item The name of the legislation, as determined by the proposer(s).
\end{enumerate}

\section{Proposition}
A bill must have at least one proposer. Only senators shall be proposers of a bill. The proposer(s) of a bill should only be those senators who were involved in the drafting, creating, or collaboration of the bill. A Senator may withdraw themselves as a proposer of the bill at any time until voting of approval begins on the bill. A Senator shall not be added to the list of proposers of the bill once legislation has been added to the docket. Should all proposers withdraw support of a bill The Senate may reserve the right with 2/3 majority to approve the bill with “Senate” as the proposer.

\section{Sponsorship}
A bill may have a list of sponsors. Any relevant person or entity may be listed as a sponsor of a bill. A sponsor of a bill should only be a person or entity who supports the bill, who has an intimate involvement with the bill, and who consents to sponsoring the bill. A person or entity may withdraw their sponsorship of the bill at any time until voting of approval begins on the bill. A person or entity shall not be added to the list of sponsors of the bill once the bill has been added to the docket.


\chapter{The Senate}

\section{Senators}
\subsection{Definition}
Any representative elected by the Student Body to the Senate shall be known as a Senator.
\subsection{Senator Requirements}
A Senator will be responsible for meeting the following requirements:
\begin{enumerate}
    \item A Senator must attend all Meetings of the Senate in a semester, provided, however, that they may accrue up to two absences.
    \begin{enumerate}
        \item Any Senator who partially attends a meaning, defined as arriving after the beginning of business or leaving prior to the completion of business, shall accrue half an absence.
        \item Any Senator who intends to be absent from a meeting shall provide written notice to the Secretary by 24 hours prior to the beginning of the meeting, and the Senator shall only accrue half an absence.
        \item Any Senator who is absent from a meeting without having provided written notice to the Secretary by 24 hours prior to the beginning of the meeting shall accrue one whole absence.
    \end{enumerate}
    \item A Senator must sit on at least one Standing Committee and meet the member requirements of that committee.
    \item A Senator must attend at least five SGA-funded events per semester, provided that those events fall in at least three different months during that semester. The Senator must report about each event via the Accountability Form within 72 hours of the event start time.
    \item A Senator must send an email to their district at least once per month.
    \item A Senator must send an email to their district within 72 hours when mandated by the Public and Communications Relations Committee members and chair.
    \item A Senator must work on initiatives and pass a monthly Status Update.
    \begin{enumerate}
        \item An initiative shall be defined as a goal that benefits the student body.
        \item The “Status Update” shall consist of:
        \begin{enumerate}
            \item Evidence that the Senator has completed or attempted to complete their initiative of the current month.
            \itemA declaration of a new initiative or redeclaration of a current initiative for the next month.
        \end{enumerate}
        \item A Senator must submit a Status Update to the Oversight Committee for approval the last seven days before the end of the month.
        \begin{enumerate}
            \item The Status Update shall require a majority affirmative vote by the members of the Oversight Committee to pass.
            \item If the Status Update fails to pass, then the Senator may re-submit their Status Update within three days of notification of their failure to pass.
            \item If a Senator fails to have their Status Update passed by the Oversight Committee after the fourth day of the month, the failure is final.
        \end{enumerate}
    \end{enumerate}
\end{enumerate}

\section{Senate Districts}
The members of each class year shall be divided into a number of groups equal to the number of Senate seats for that class. These groups shall be equal or approximately equal in size and have randomly determined constituents. Each Senator shall be assigned one of these groups, hereafter referred to as “Districts”, at the beginning of each academic year. Freshmen Senators shall be assigned Districts upon taking office. If a vacancy of a Senate seat is to occur, the Senator who fills the vacancy shall be assigned the same District as the person who previously held the seat. Senators shall be responsible for regular communication with the constituents of their District and shall serve as a means of the constituents of their District sharing their concerns and opinions with the SGA as a whole. The Secretary shall be responsible for organizing and assigning Districts. The term “class year” shall be defined by the Constitution of the SGA.

\section{Elections of the Senate}

\subsection{Frequency}
Elections of the Senate shall occur once per academic year. 

\subsection{Nominations}
Nominations shall open two weeks before the constitutionally mandated election 
period and shall close one week thereafter. A prospective candidate must 
deliver a petition with signatures from 50 students within their class year to 
the Secretary in order to become nominated.

\subsection{Verification} \label{sec:election_verification}
The Secretary, in conjunction with the Office of Student Life, shall then have 
one week to determine if the nominees meet the eligibility requirements to be 
a Senator as defined by Institute policy or established by resolution of the 
Senate. 

\subsection{Election}
The appropriate administrators shall be directed to enable the voting system 
immediately after the completion of the verification period. The election 
shall remain open for one week. The option to reopen nominations shall always be included in the election poll. In the event that any nominees garner less votes than the option to re-open nominations, those nominees shall not be elected in that election cycle. Of the remaining nominees, those receiving the most votes will fill the vacant seats. Any nominee who failed to be elected shall still have the ability to run in future elections. The elected Senators shall assume office at the beginning of the next meeting of the Senate.

\subsection{Vacancy}
In the circumstance of a vacancy of a Senate seat or receipt by the Secretary of a
Senator’s resignation, a vacancy election shall be held. Once the Election
Committee recognizes the need for a vacancy election, the nomination period must
begin within five days. This period after the acknowledgment of the resignation or
vacancy and before the nomination period begins shall be used by the Public
Relations Committee to prepare a publicity campaign, by the Secretary to contact
the appropriate administrators to set up a vacancy election, and by the Election
Committee to determine the election timeline and announce it to the Senate who
shall announce it to the Student Body. The nomination period will last for an
amount of time determined by the Election Committee of no less than three days
and no more than four days. There shall be a verification period immediately following the nomination period which shall last for an amount of time determined
by the Election Committee of no less than one day and no more than two days. The
appropriate administrators shall be instructed to open the voting period, which shall
last for three days, immediately following the closure of the verification period.
The results must be certified by the verification procedure outlined in Section
7.3(E). Elected Senators shall assume office once verified by the Election
Committee. If the preliminary vacancy election procedure ends with Senate seats left vacant,
the vacancy election procedure shall recommence. These subsequent vacancy
elections periods shall begin no more than three days following the end of the the
previous vacancy election period. There shall be no more than three repetitions of
the vacancy election procedure. Members of the Student Body shall have one vote
per vacancy, but may not vote for the same candidate for more than one vacant
seat.

If a vacancy is to occur while the Senate is not in session, the vacancy election
procedure shall commence once the Senate is in session.

\section{Senate Resignations}
In order to resign, a Senator must submit a formal resignation to the Secretary and
Speaker. By submitting a formal resignation, a Senator immediately forfeits their Senate seat and shall not be able to resume their position
unless elected to the Senate by the Student Body in a general or vacancy election. Once submitted, a formal resignation may not be withdrawn.
\section{Meetings of the Senate}

\subsection{Quorum of the Senate}
Quorum of the Senate shall be defined as two-thirds of all Senators. Any 
Senator, for whom a motion of impeachment is on the docket shall not be counted as a 
Senator toward quorum. At a meeting where quorum does not prevail, but where a simple majority is present, quorum can be overruled at the discretion of the President, and the 
concurrence of the Speaker. 

\subsection{Prior Announcement}
All Senate meetings must occur at regularly scheduled intervals as
resolved by the Senate, or be announced one week prior to the meeting.

\subsection{Speaker}
\begin{enumerate}
    \item The Speaker shall chair all meetings of the Senate. 
    \item The Speaker shall be elected to a term coinciding with the current session of the Senate. The Speaker shall be elected during the Senate’s first meeting of a new session, which shall directly follow the last meeting of the previous session. The first order of business during this meeting shall be to elect a Speaker. The Speaker of the previous session shall chair this meeting until the new Speaker has been elected. Once a motion to elect a Speaker has been entertained, the Speaker of the previous session of the Senate shall accept nominations from the floor. Only individuals who have served at least one semester in the capacity of an Official of the SGA shall be eligible to be nominated. Only Senators may nominate a Speaker, and each nomination must be seconded by another Senator.The winning nominee shall, in a vote by secret ballot, accumulate a simple majority of the votes of the Senators present.
    \item When in the circumstance that a Senator is chosen as Speaker, they 
          may retain their position as Senator, but must abstain from voting 
          on any business unless: the Senate is equally divided or when not 
          presiding in cases of impeachment trials. 
        \begin{enumerate}
            \item In this circumstance, the Speaker shall retain the right to 
                  speak, but may not propose any motion to the Senate. 
        \end{enumerate}
    \item The Speaker may not miss more than two meetings over the course of 
          the semester. 
        \begin{enumerate}
            \item In cases where a Senator also serves as Speaker, the Speaker 
                  shall be counted toward quorum. In cases where a non-Senator 
                  serves as Speaker, the Speaker may not be counted toward 
                  quorum. 
        \end{enumerate}
    \item Upon a resignation from the office of Speaker, the Senate shall not accept the Speaker’s resignation except pending the election of a new Speaker, wherein the new Speaker will take office immediately following the acceptance of the former Speaker's resignation.
\end{enumerate}

\subsection{Speaker Pro-tempore}
In the event the Speaker cannot preside over the Senate meeting, the duty 
shall fall to the Speaker Pro-Tempore. The Speaker Pro-tempore shall be chosen 
by the Speaker. 

\subsection{Secretary Pro-tempore}
In the event that the Secretary is unable to attend all or part of the 
meeting, the Secretary Pro-Tempore from the Senate will fill in to take the 
minutes for the meeting, or until the Secretary returns. The Secretary 
Pro-tempore shall be chosen by the Secretary

\subsection{Roberts Rules of Order}
Meetings of the Senate shall be conducted in accordance with the Constitution 
of the SGA and these bylaws. Business conducted during meetings shall refer to 
the latest edition of Roberts Rules of Order when necessary. Roberts Rules of 
Order shall not supersede the Constitution of the SGA or these bylaws. 

Should a motion not be in order, it is the duty of the presiding chairperson 
to determine if the motion and all motions that depend upon it can be feasibly 
unwound. 

The Parliamentarian may review any motion and advise the Senate to withdraw 
the motion, prior to further debate, if the motion is unconstitutional or 
conflicts with these Bylaws. 

\subsection{One-Third Abstention}
Should more than one-third of the senators present abstain on a vote on a 
debatable motion, the item of business shall be returned to the Senate for 
further debate. 

\subsection{Debate}
In order to promote free and open debate, a motion to close debate shall not 
be entertained within 5 minutes of the start of debate unless approved by 
unanimous consent. The chair shall exhaust all comments from Senators before 
progressing to comments from the Cabinet, which shall be exhausted before 
progressing to comments from the public, in a cyclical fashion until every 
comment has been exhausted. However, the chair may, at his discretion, 
determine that a comment from an individual is in response to another comment 
made and call on them out of order. 

\subsection{Censure}
The Senate may vote to censure a Senator if the Senate deems that the Senator 
is speaking irresponsibly. A vote of seventy-five percent of the Senate is 
required. 

Should a Senator be censured, they will surrender their right to speak until 
the business item has been resolved. They shall not lose their right to vote 
on the business item. 

\section{Sessions of the Senate} \label{sec:senate_sessions}

\subsection{Definition}
The session of a particular Senate shall begin at the start of the Senate 
meeting following the conclusion of Senate elections, and end at the start of 
the next session of the Senate. A session of the Senate is separate from the
parliamentary session, which shall remain defined as the default as adopted
from the Senate’s parliamentary authority.

\subsection{Senate Number}
Each session of the Senate shall be numbered, beginning with the first session 
of the Student Council in April 1913. The Senate of the 2012-2013 school year 
shall be the ninety-ninth Senate. 

\subsection{Recess}
The Senate may declare, by two-thirds resolution, a period in which it shall 
be considered ``in recess.'' No business may be conducted nor meetings held by 
the Senate while it is in recess, except that a meeting may be held to 
consider a resolution to come back from recess ahead of schedule. 

\subsection{Reduced Session}
During any period in which either class is not held or less than half of the 
Student Body is registered for classes, and the Senate is not in recess, the 
Senate shall be considered to be ``in reduced session.'' Attendance to any 
Senate meetings shall not be mandatory during reduced session. 

\subsection{Full Session}
The Senate shall be considered ``in full session,'' or simply ``in session,'' 
at any time that it is neither in recess nor in reduced session. 

\section{Standing Motions}
Standing motions of the Senate may be made by resolution of the majority of 
the Senate and approval of the President. No standing motion shall be 
inconsistent with or supersede the bylaws or Constitution of the SGA. Standing 
motions shall only have effect until the close of the current session of the 
Senate. No standing motion shall be construed to grant additional powers to 
recognized organizations or the SGA. 

\section{Financial Powers}
The Senate shall have the power to transfer money between SGA accounts and 
Campus Organization accounts. The Senate may not transfer money out of the 
Discretionary Fund held by the Cabinet. The Senate shall have the power to 
spend money out of the main SGA account. However, any motion to spend money must be approved by a simple majority of the Senate where quorum prevails. 

\chapter{The Cabinet}

\section{Presidential Elections}

\subsection{Frequency}
The Presidential Election shall occur once per academic year, during the 
constitutionally mandated election period. 

\subsection{Nominations}
Nominations shall open three weeks before the constitutionally mandated 
election period and shall close one week thereafter. A prospective candidate 
must submit a petition with signatures from five percent of the Student Body 
in order to become nominated. Nominations shall be submitted to the Secretary, 
who shall immediately forward them to the election committee. 

\subsection{Verification}
The election committee, in conjunction with the Office of Student Life, shall 
then have one week to determine if the nominees meet the eligibility 
requirements to be a member of the Cabinet established by resolution or 
standing motion of the Senate or by Student Life policy. 

\subsection{Election}
The appropriate administrators shall be directed to enable the voting system 
immediately after the completion of verification. The election shall remain 
open for one week. The President \& Vice President of Operations Elect shall be announced at 
the beginning of the next meeting of the Senate. They shall assume office and 
be sworn in at noon on the first day of the Spring Semester, in accordance 
with the Constitution of the SGA. 

\section{Archiving}
The Secretary will oversee the archiving of all writs, resolutions, bills, 
proclamations, dockets, addendums, and bylaws in the SGA archives. The Secretary will also be 
responsible for archiving the constitutions, and bylaws of all RSOs, as 
outlined in \ref{sec:document_archival}, and any other material presented to 
the Cabinet by a duly authorized officer of an organization. These archives 
will be available to the Student Body electronically.

\section{Meetings of the Cabinet}
The Cabinet shall meet during the constitutionally mandated period. The 
Cabinet may adopt its own rules for how Cabinet meetings shall run.

\section{Senate Meeting Agendas}
The Secretary will conduct the formation of each and every Senate agenda, 
referred to as ``Docket''. The format will consist of: 
\begin{enumerate}
    \item In order, the following: ``Docket for the,'' the number of the 
          current session of the Senate, ``Senate of the Student Government 
          Association,'' and the date of the Senate meeting at which the 
          agenda is addressed.
    \item A discussion period referred to as “Public Forum” in which members of the Student Body who are not Officials of the SGA in attendance may discuss items they wish to bring to the attention of the Senate.
    \item Reports of the Speaker, President, and each member of the Cabinet.
    \item Committee reports of each committee of the SGA, each sub-committee 
          of the Committee on Student Interests, and the reports of any other 
          sub-committees compelled by standing motion of the Senate to be 
          produced at a Senate meeting. 
    \item Items currently lain on the table by the current session of the 
          Senate. 
    \item The order of business to be conducted, which shall begin with Old 
          Business and conclude with New Business. Within each of order of 
          business section, the docket will consist of: 
        \begin{enumerate}
            \item Any bills to be presented to the Senate during the meeting, 
                  in the form of a motion to approve the bill with a Senate 
                  sponsor of the bill listed as the individual expected, but 
                  not required, to make the motion. 
            \item Any other resolutions to be presented to the Senate during 
                  the meeting, in the form of a motion to approve the 
                  resolution with a Senate sponsor of the resolution listed as 
                  the individual expected, but not required, to make the 
                  motion. 
            \item Any motions regarding financial items outside of the scope 
                  of a resolution, such as approval of a budget or 
                  constitution. 
            \item Any business regarding impeachments either brought by a 
                  member of the Senate or necessitated by the bylaws. 
            \item An indicator that any other motions will be entertained from 
                  the Senate before progressing on to other sections of the 
                  order of business. 
        \end{enumerate}
    \item Any other items required to be placed on the Docket by these bylaws 
          or by standing motion of the Senate.
    \item A discussion period following the completion of all business during which the Senate, Cabinet, and any other members of the Student Body in attendance may discuss any issues they see fit for discussion at a Senate meeting.
\end{enumerate}

\section{Presidential Discretionary Fund}
There shall be an amount of money of 2000 reserved. This amount of money shall be called the “Presidential Discretionary Fund. The President shall be authorized to spend money from this fund without the involvement of the Senate. At no time shall the fund exceed ten times the Student Activity Fee. 

\chapter{Oversight}

\section{The Oversight Committee}

\subsection{Members} \label{sec:oversight_membership}
The Oversight Committee shall be comprised of the Speaker and one Senator from 
each class. In accordance with the Constitution of the SGA, Senators on the Oversight Committee representing each class shall be elected by the entire Senate and approved by the Cabinet. Senators shall serve on the Oversight Committee for a term coinciding with the session of the 
Senate. The Freshmen Oversight Committee representative shall be elected once members of the freshmen class have been elected to the Senate. The Speaker of the Senate shall designate a member of the Oversight Committee to collaborate with the Public Relations Committee on the upkeep of the District Distribution List, as outlined in Section 7.4.(A).

If a vacancy of a seat on the Oversight Committee occurs, aside from the position of Speaker, the Senate shall elect a Senator from the same class as the individual who previously held the seat to fill the vacancy. Senators appointed to fill a vacancy shall serve until the next session of the Senate. 

If a Senator who is serving on the Oversight Committee is brought for 
disciplinary action, the Senator shall be removed from the Oversight Committee 
until their disciplinary action is resolved in the Senate. The Speaker shall 
appoint a Senator from the same Class to serve on the Oversight Committee as 
Pro tempore, until the disciplinary action is resolved. 

\subsection{Meetings}
The Oversight Committee shall meet at the request of a Senator, the Speaker, 
the Vice President of Operations or when constitutionally mandated. The Chairperson of the 
Oversight Committee shall be the Speaker, except in cases where the Speaker is 
brought for impeachment. 

Quorum for a meeting of the Oversight Committee shall be obtained when 
seventy-five percent of its members are present. Should a Senator miss two 
Oversight Committee meetings over the course of a Semester, they shall forfeit 
their seat on the Committee. The seat shall be filled according to 
\ref{sec:oversight_membership}. 

The Vice President of Operations may attend committee meetings, to represent the interests 
of the Cabinet, but may not vote on any matters. The Oversight Committee will 
review all writs of impeachment or other disciplinary actions brought before 
Officials of the SGA. 

\subsection{Disciplinary Action} \label{sec:disciplinary_action}
When it deems necessary, the Oversight Committee may bring disciplinary action upon an Official of the SGA, where a simple majority of the Oversight Committee shall pass a writ of action. If this action is passed, the accused Official of the SGA shall be brought up before the Senate for an impeachment trial. A two-thirds majority of the Senate is required to approve the disciplinary action, unless otherwise defined in these bylaws. The Oversight Committee may only consider the disciplinary actions outlined in \ref{sec:disciplinary_action}. 
\begin{enumerate}
    \item Charge an Absence -- If deemed necessary, the Oversight Committee 
          may propose that a Senator be charged an absence.
    \item Removal of Title -- If deemed necessary, the Oversight Committee may 
          propose that a Senator surrender their position (i.e. Committee 
          Head, Sub-Committee Head, Speaker, etc.) 
    \item Impeachment -- A Writ of Impeachment of an official of the SGA may 
          only be considered when constitutionally sanctioned. If the 
          Committee passes a Writ of Impeachment, a bill of Expulsion shall be 
          brought before the Senate to expel the Official. The approval of 
          seventy-five percent of the Senate is required to expel any 
          official from office. 
\end{enumerate}
As mandated by the constitution, impeachment shall be defined ``as an 
accusation of an Official for failure to fulfill the duties for which they are 
responsible''. 

\section{The Parliamentarian}

\subsection{Appointment}
Upon assuming office, the President shall appoint a member of the Student Body 
to the office of Parliamentarian of the SGA. The Parliamentarian is not part 
of the Cabinet, but shall be its own position within Constitutional Oversight 
of the SGA. 

\subsection{Responsibilities}
It shall be the responsibility of the Parliamentarian to be knowledgeable the 
Constitution of the SGA, of these Bylaws, of the addendums to these bylaws, and of Roberts Rules of Order. 

It shall be the responsibility of the Parliamentarian to review any motion 
brought before the Senate, prior to its discussion, to determine if the motion 
concurs with the Constitution of the SGA, these Bylaws and Roberts Rules of 
Order. 

It shall be the responsibility of the Parliamentarian to assist the Constitution and Bylaws Committee with the upkeep of the Constitution of the SGA and these Bylaws, as outlined in Section 7.2(D).

\subsection{Pro-tempore Responsibilities}
The Parliamentarian shall serve as Chairperson of the Oversight Committee only 
in cases where the Speaker is brought for impeachment. In such cases, the 
Parliamentarian may only vote in cases of a tie. 

\chapter{Recognized Student Organizations} \label{sec:rsos}

\section{Recognized Student Organization}

\subsection{Definition}
A Recognized Student Organization, herein referred to as an ``RSO'', shall be 
defined as a Campus Organization that has obtained recognition from the SGA.

\subsection{Procedure}
In order to obtain recognition, eligible Campus Organizations must follow the procedures outlined in The New Organization Process.

\subsection{Duration}
A Campus Organization shall be considered an RSO during the semester in which
it has successfully completed the procedures outlined in The New Organization Process and received full RSO status from the SGA, and any following semesters, even when the Senate is not in full session.

\section{Inactive Organizations}

\subsection{Definition}
An Inactive Organization shall be defined as any approved RSO that fails to 
register with the Cabinet \& the Office of Student Life.

\subsection{Determination}
If an RSO becomes inactive, the Vice President of Student Interests or the Office of Student Life reserves the right to approach members of the organization, to determine if the 
organization is still active outside of the auspices of the SGA.

\subsection{Inactive}
If the number of voting members required for quorum of the organization cannot 
be located by the Vice President of Student Interests, or a majority of located voting members do not affirm that the organization is currently active, then the Vice President of Student Interests shall publicize a new emergency election and shall open nominations for a new executive board of the organization.

\subsection{Inactive or Unable to Reorganize}
Should the organization affirm it is an active RSO, but a new executive board 
is not formed, or the new executive board does not meet the requirements for 
recognition within one week after they take office, the organization shall 
return all SGA funded items and equipment and become ineligible for 
recognition by the SGA until it reorganizes following the process outlined in the New Organization Process addendum to these bylaws.

\section{Nonpartisan Funding}
No Campus Organization may use SGA-provided funds to promote, except as 
provided for by the rights of a free press, any candidates for any election. 
Campus Organizations may promote candidate awareness by providing unbiased 
information on each candidate, or by promoting every candidate equally.

\section{Public Forum}
The publications and content distribution mechanisms of all organizations 
shall be considered a ``public forum.'' Individuals shall be liable for their 
own statements when conveyed by organizations.

\section{Officers and the Executive Board}

\subsection{Unified Elections}

\paragraph{Officer Terms}
Officers of an RSO will serve a term of one-year or until the next
Unified Elections, and until their successor is elected, or a lesser
term if specified in the RSO’s Constitution. Officers of an RSO will
take office immediately following their election, unless otherwise
specified in the RSO’s Constitution.

\paragraph{Timeline}
Unified Elections shall occur once every academic year. The first full
week of school following the University-defined Spring Recess shall be
the Unified Elections period. The last week of school preceding Spring
Recess shall be the Unified Nominations period. An RSO must accept
officer nominations during the Unified Nominations period, submit all
nominations to the Office of Student Life immediately thereafter, and
then hold a special meeting for elections of all officers during the
Unified Elections period, in which any ballot cast for an individual
whose nomination was not previously approved by the Office of Student
Life will be considered invalid.

\paragraph{Custom Elections}
An RSO may define in their Constitution a custom period for their
Unified Elections, provided there is at least one week in between the
Unified Nominations period and the Unified Elections period, and that
the elections process complies with all other rules specified in these
bylaws, especially above.

\paragraph{Changes}
Any changes in an RSO’s officers outside of the standard Unified
Elections period (including RSOs that conduct Unified Elections during
a custom time period) may only occur after notification of the Vice President of Student Interests and after verification by the Office of Student Life of any individuals intending to become an officer.

\paragraph{Vacancies}
Unless the RSO’s Constitution specified otherwise, when a vacancy in
an RSO officer position occurs, a special election process shall be
held following the same procedures as Unified Elections, except that
the Unified Nominations period will begin the first full week of class
following the first meeting after the vacancy exists, and the Unified
Elections period will begin the second full week of class following
the end of the Unified Nominations period.

\subsection{Executive Board}

\paragraph{Definition}
Every RSO will have an Executive Board, consisting only of officers of
the RSO, and including all officers of the RSO unless otherwise
defined in the RSO’s Constitution. A majority of the RSO’s Executive
Board must consist of elected officers.

\paragraph{Funding}
Only a majority of the Executive Board, or higher if specified by the
RSO by Constitution, bylaw, or standing rule, shall have the power to
authorize spending of the RSO’s SGA-allocated funds, provided such
spending is in compliance with the RSO’s budget as approved by the
SGA.

\paragraph{Appointed Officers}
Any officers of an RSO that are not elected by the RSO’s general body
must be appointed by a majority, or higher if specified by the RSO by
Constitution, bylaw, or standing rule, of only those members of the
Executive Board who themselves were elected by the RSO’s general body.

\section{Financing of Campus Organizations}

\subsection{Fund Allocation}
A fund allocation is any amount of money that the Senate transfers between SGA accounts and Campus Organization accounts. The Campus Organization will be responsible for following all policies outlined in The Financial Policy Addendum to these bylaws in order to receive and maintain fund allocations from the SGA.

\section{Votes of Executive Boards}
Any meeting not open to the entire Student Body shall be considered to be held 
in executive session. The executive board of an RSO may only hold exclusive votes on items that are left to solely its discretion that are outlined in the RSO’s constitution. The voting membership of the general body of an RSO shall have the right to vote on all items not left solely to the discretion of its executive board.

\section{Overruling of Quorum or Voting Membership}
Should an RSO become unable to conduct business due to its inability to achieve quorum or the necessary amount of eligible voting members at any of its meetings as a result of an ineffective definition of quorum or voting membership in its constitution or bylaws, the RSO’s quorum or definition of voting membership may be temporarily overruled by the Vice President of Student Interests. The Vice President of Student Interests shall mandate a more suitable, temporary definition of quorum or voting membership for the RSO that allows for business to be conducted at its meetings. This temporary definition of quorum or voting membership shall only last until the RSO has voted on the business it initially sought to conduct, amended its definition of quorum or voting membership in its constitution or bylaws to be more suitable, and had any changes to its constitution reviewed by the Constitution and Bylaws Committee and approved by the Senate. This process shall not occur unless the RSO submits evidence to the Vice President of Student Interests of having held at least one special meeting designated solely for the purpose of conducting business and having notified the entirety of its voting membership of this meeting at least one week prior.

\chapter{Committees}

\section{Types}

\subsection{Standing Committees}
Standing Committees shall be defined as permanent committees of the SGA which shall exist to perform certain duties on an ongoing basis. In order for a committee of the SGA to be classified as a Standing Committee, it must have its name and purpose outlined in Article VII of these Bylaws. Standing Committees may have further rules and responsibilities outlined in other governing documents of the SGA, such as the Constitution and addendums to these bylaws.

\subsection{Ad Hoc Committees}
An Ad Hoc Committee shall be defined as a committee of the SGA which shall exist only temporarily to perform certain duties. No governing documents of the SGA shall refer to any Ad Hoc Committee by name.

\section{Chairs}
Unless otherwise specified in any governing documents of the SGA, each committee shall have one chair, which shall be a Senator.

\section{Members}
Unless otherwise specified in the definition of the body, every committee 
shall allow any member of the Stevens community to be a member of the 
committee.

Every Senator, with the exception of the Speaker if they are a Senator, must serve on at least one Standing Committee. Participation on the Committee on Student Interests, Election Committee, and Festival Committee shall not satisfy this requirement. 
Senators shall be assigned to Standing Committees by the Vice President of Operations based on each Senator’s interest and the need for Senators on each Committee. Senators must meet attendance requirements of each Standing Committee on which they serve as well as perform all duties assigned to them by the Committee Chair, or be eligible for disciplinary action as defined in \ref{sec:disciplinary_action}.

\section{Positions}
The chair of a committee shall be permitted to create or remove officer 
positions in the committee at will, and appoint members to those positions, 
unless otherwise specified in the definition of the committee.

\section{Sub-Committees}
Standing sub-committees shall exist only within the Committee of Student 
Interests, and are further defined in \ref{sec:csi_subcommittees}.

Ad-hoc sub-committees may exist at the discretion of the parent committee. The 
chairs of ad-hoc sub-committees shall be appointed by the chair of their 
parent committee.

\section{Prohibition on Committee Bypass}
Committees may be charged with various duties otherwise left to the entire 
Senate, either by the bylaws or by administrative assignment. Committees may 
also be charged with the duty of prior review before a motion may be presented 
to the Senate. A motion before the Senate that bypasses these duties of a 
committee shall not be entertained, unless the committee has already been 
presented with the issue and has either not resolved it in a timely manner, or 
has ruled in a manner unacceptable to the person or organization presenting 
the issue.

\chapter{Standing Committees}

\section{The Committee on Student Interests} \label{sec:csi_representative}

\subsection{Purpose}
The Committee on Student Interests (``CSI'') shall be the forum for the 
exchange of information between RSOs and the SGA.

\subsection{Composition}
The CSI shall be composed of the Vice President of Student Interests, one representative from each RSO as designated by the executive board of each organization herein as referred to as the ``RSO Representative''. Other members may be added or removed at the discretion of the Vice President of Student Interests.

\subsection{Standing Sub-Committees of the CSI} \label{sec:csi_subcommittees}
The Committee shall be broken into Standing Sub-committees as determined by 
these bylaws. Each sub-committee will be made up of one RSO Representative from each RSO within the sub-committee, unless otherwise 
specified in these bylaws. RSOs shall be assigned to appropriate sub-committees at the discretion of the Vice President of Student Interests.

The chairperson of each sub-committee (known as the Sub-commitee Head) shall be elected by plurality from among the subcommittee members. The Sub-committee Head is not required to be a Senator. Elections shall occur in accordance with RSO Unified Elections policies as specified in Section 5.6.

Each sub-committee shall be required to hold at least one official
meeting per month, during the Fall and Spring Semesters. Seventy-five
percent of sub-committee members must be present to obtain quorum. The
Sub-committee Head shall be required to notify all members of the
sub-committee about an official meeting two weeks prior to that
meeting.

\subsubsection{Ethnic Student Council}

\paragraph{Purpose}
The purpose of the Ethnic Student Council, herein referred to as ``ESC'', 
shall be to coordinate efforts of the ethnic RSOs, to unite the ethnic RSOs, 
and to promote and celebrate the cultural diversity among the Student Body.

\paragraph{Officers}
The Executive Board of the ESC shall be comprised of the President, Vice 
President, Treasurer, Secretary, and Social Chair. ESC Executive Board members 
may not serve on the Executive Board of any member organizations. Officers of 
the ESC shall be determined by following the Unified Elections procedure for 
RSOs.
\subparagraph{President}
It shall be the responsibility of the ESC President to act as the official
representative of the ESC. The President shall preside over all meetings of 
ESC.
\subparagraph{Vice President}
It shall be the responsibility of the ESC Vice President to perform all duties 
of the President in the absence of the President.
\subparagraph{Treasurer}
The ESC Treasurer shall manage all financial transactions and records of the
ESC. The Treasurer shall perform all duties assigned by the President.
\subparagraph{Secretary}
The ESC Secretary shall maintain accurate and complete minutes of all
meetings. The Secretary shall prepare all correspondence of the Committee.
\subparagraph{Social Chair}
The ESC Social Chair shall maintain a calendar of all happenings within the 
ESC and its member organizations.
\subparagraph{Web Chair}
The ESC Web Chair shall work with the ESC Social Chair in scheduling and planning events for organization under ESC.
\subparagraph{Publicity Chair}
The ESC Publicity Chair shall design publicity (flyers,images, logos, invitations, t-shirts, etc.), send prints to publicity, distribute flyers, and work with the ESC Social Chair and ESC Web Chair for event publicity.

\paragraph{Requirements of Committee Members}
Two representatives from each recognized ethnic RSO shall have voting 
membership within the Committee.

\paragraph{Advisor}
A Faculty or Staff Advisor shall be elected from eligible faculty and staff, 
as determined by the Office of Student Life.

\subsubsection{Professional Societies Council}

\paragraph{Purpose}
The purpose of the Committee shall be to coordinate efforts of Professional 
RSOs, to coordinate efforts between the administration and Professional RSOs, 
to unite Professional RSOs, to coordinate national efforts (e.g. National 
Engineers' Week), to encourage membership in Professional RSOs, and to 
stimulate interest in the goals of Professional RSOs among individuals who are 
not majoring in the targeted field.

\paragraph{Responsibilities}
The Committee shall consider for acceptance, hold, and execute such
responsibilities as delegated to the Committee from the member Professional 
RSOs and the SGA.

\subsubsection{Media Board}

\paragraph{Purpose}
The purpose of the Media Board shall be to coordinate efforts of the Media 
RSOs, to collectively protect the rights of a free press on campus, to unite 
the Media RSOs, to encourage membership in Media RSOs, and to stimulate 
interest in the products of the Media RSOs.

\paragraph{Responsibilities}
The Committee shall consider for acceptance, hold, and execute such
responsibilities as delegated to the Committee from the member Media RSOs and 
the SGA.

\subsubsection{Special Interests Council}

\paragraph{Purpose}
The purpose of the Special Interests Council shall be to coordinate efforts of 
the Special Interest RSOs, to unite the Special Interest RSOs, and to 
encourage membership in Special Interest RSOs.

\paragraph{Responsibilities}
The Committee shall consider for acceptance, hold, and execute such 
responsibilities as delegated to the Committee from the Special Interest RSOs 
and the SGA.

\subsubsection{Club Sports Council}

\paragraph{Purpose}
The purpose of the Committee shall be to coordinate efforts of the Club 
Sports, to coordinate efforts between the administration and the Club Sports, 
to unite the Club Sports, to encourage membership in the Club Sports, and to 
stimulate interest in the Club Sports.

\paragraph{Responsibilities}
The Committee shall consider for acceptance, hold, and execute such 
responsibilities as delegated to the Committee from the member Club Sports and 
the SGA.

\paragraph{Advisor}
An advisor shall be appointed by the Director of Athletics.

\subsubsection{Arts \& Music}

\paragraph{Purpose}
The purpose of the Committee shall be to coordinate efforts of Arts and Music 
RSOs, to coordinate efforts between the administration and Arts and Music 
RSOs, to unite Arts and Music RSOs, and to stimulate interest in the 
programming created by Arts and Music RSOs.

\paragraph{Responsibilities}
The Committee shall consider for acceptance, hold, and execute such 
responsibilities as delegated to the Committee from the member Arts and Music 
RSOs and the SGA.

\subsubsection{Recreation}

\paragraph{Purpose}
The purpose of the Recreation Committee shall be to coordinate efforts of the 
Recreational Sporting RSOs, to coordinate efforts between the administration 
and Recreational Sporting RSOs, to unite, encourage membership, and to 
stimulate interest in Recreational Sporting RSOs.

\paragraph{Responsibilities}
The Recreation Committee shall consider for acceptance, hold and execute such
responsibilities as delegated to the Committee from its member Recreational
Sporting RSOs and the SGA.

\subsubsection{Electronics and Gaming}

\paragraph{Purpose}
The purpose of the Electronics and Gaming Sub-committee shall be to coordinate efforts of the Electronics and Gaming RSOs, to coordinate efforts between the administration and Electronics and Gaming RSOs, and to unite, encourage membership for, and simulate interest in Electronics and Gaming RSOs.

\paragraph{Responsibilities}
The Sub-committee shall consider for acceptance, hold, and execute such responsibilities as delegated to the Sub-committee from the Electronics and Gaming RSOs and the SGA.

\subsubsection{Service}

\paragraph{Purpose}
The purpose of the Service sub-committee shall be to coordinate
efforts between the administration and the Service RSOs, to encourage
membership in Service RSOs, to advertise opportunities and encourage
campus to take action and do something about national and global
issues, and to provide a network for the Service RSOs to promote
educational seminars, events, or community service to benefit their
specific organization, but more importantly, the community.

\paragraph{Responsibilities}
The Committee shall consider for acceptance, hold, and execute such
responsibilities as delegated to the Committee from the member Service
RSOs and the SGA.

\paragraph{Advisor}
A Faculty or Staff Advisor shall be elected from eligible faculty and
staff, as determined by the Office of Student Life.

\subsubsection{Religious and Faith Based}

\paragraph{Purpose}
​The purpose of the Religious and Faith Based Subcommittee shall be to coordinate efforts of the religious and faith based RSOs, to unite these RSOs, and to celebrate and educate about the religious diversity among the Student Body.

\paragraph{Responsibilities}
​The Committee shall consider for acceptance, hold, and execute such responsibilities as delegated to the Committee from the Religious and Faith Based RSOs and the SGA.

\subsection{Meetings of the CSI}
Meetings of the CSI, which shall be called ``Leadership Connects'', shall be conducted at least once per semester and must be announced at least two weeks prior to the meeting. The purpose of these meetings shall be to share updates on behalf of the SGA and the Office of Student Life with RSOs as well as offer leadership advice to RSO executive boards.

The Vice President of Student Interests may, at their discretion, suspend the funding of any RSO that fails to send a representative to two consecutive meetings of the CSI.

\subsection{The Sub-committee Head Council}
There shall be an organization known as the Sub-committee Head Council that shall be composed of every Standing Sub-committee Head. The Sub-committee Head Council shall meet monthly to hold general discussion regarding RSOs. The Sub-committee Head Council shall also be responsible for approving new RSOs, as outlined in the New Organization Process addendum to these bylaws.

\section{Constitution and Bylaws Committee}

\subsection{Purpose}
The purpose of the Committee is to ensure the Constitution and Bylaws of the SGA remain up
to date, as well as serve as a resource for Senators wishing to ensure legislation presented to the Senate is properly written and formatted. The Committee shall also be responsible for reviewing the constitutions and bylaws of the various RSOs of Stevens Institute of Technology.

\subsection{Review of Constitutions}
The Committee is charged with reviewing any and all constitutions of organizations seeking RSO status before they are presented to the Senate, as well as any amendments to an existing RSO’s constitution approved by its voting membership. The Committee will insist that any presented constitutions be in compliance with the various rules for RSOs defined in 
\ref{sec:rsos}, be grammatically correct and coherent, and offer other recommendations to ensure for the longevity of the organization. The Committee shall ensure the constitution of an RSO, at minimum, includes the name of the organization, a stated purpose, voting membership requirements, a list of executive board positions and associated responsibilities, a stated election period for officers of the executive board, a definition of quorum, and a chaos clause.

\subsection{Assistance and Review of Bylaws}
The Committee shall, upon request, assist organizations in creating appropriate bylaws for 
the governance of their club, and shall bring to the Senate's attention any 
organization's bylaws that the Committee feels is in conflict with the 
organization's constitution, the Constitution of the SGA, or these bylaws.

\subsection{Review of Consitution and Bylaws}
The Committee shall meet at least once every Fall Semester within the first month of classes to review the Constitution of the SGA and at least once every Spring Semester within the first month of classes to review the Bylaws of the SGA. If the Committee deems that revisions be necessary to the Constitution or Bylaws during its review of either document, it shall, as a committee, present a bill to the Senate within that semester which shall propose revisions which remove any outdated policies and add any new policies not yet codified in each respective document. These bills shall be titled “Recommendations of the Constitution and Bylaws Committee for the Upkeep of”, followed by either “the Constitution of the Student Government Association” or “the Bylaws of the Student Government Association”, followed by “for the”, followed by the name of the semester in which the bill is being proposed, followed by “Semester”. In order to ensure the Committee’s recommendations are appropriate, bills of this nature shall not be presented to the Senate unless sponsored by the Parliamentarian.

\subsection{Review of Legislation}
The Committee shall, upon the request of a Senator, be charged with reviewing any legislation intended to be presented to the Senate. The Committee shall ensure that the legislation meets the requirements set forth in \ref{sec:bill_content}, is grammatically correct and coherent, does not conflict with the Constitution or Bylaws of the SGA where not intended, and offer other objective assistance to ensure the wording of the legislation is aligned with its purpose.

\subsection{Document Archival} \label{sec:document_archival}
The Committee shall be charged with assisting the Secretary in archiving the 
constitutions and bylaws of RSOs, and the constitution and bylaws of the 
Student Government Association.

\section{Election Committee}

\subsection{Purpose}
The Committee shall be responsible for administrating all active elections for 
the Cabinet and Senate. The Election Committee shall be responsible for organizing a debate between the candidates/nominees for the Freshman Senator Elections, as well as for the
Presidential Elections. The debates must fall within 2 weeks’ time of the end of the
nomination period. The election committee shall announce the dates of
nomination, dates of elections, and date of the debate to the senate. The senate
shall announce these election rules to the student body. 

\subsection{Requirements of Committee Members}
The term class year shall be interpreted as defined by the Constitution of the 
SGA. In accordance with the Constitution of the SGA, the Secretary shall chair the Election Committee. The Secretary shall appoint one member of each grade from
the Honor Board to serve on the election committee. The sophomore, junior, and
senior representative of the election committee shall be appointed within the first 2
weeks of the academic year. The freshman representative of the election shall be
appointed within 2 weeks of the freshmen honor board elections.

\subsection{Conflict of Interest}
Should a member of the Student Body run for a Cabinet position, be promised an 
appointed Cabinet position, or be running for any Senate seat, the person 
shall become ineligible for membership on the Committee.

\subsection{Improprieties and Errors}
Should any candidate/nominee fail to abide by the rules and procedures of the
election s outlined in The Election Procedures and Guidelines Addendum, the infraction shall be brought up to the Election Committee. Should the Election Committee deem an infraction valid, they shall then decide whether the infraction is Minor or Severe. The Election Committee shall have the right to issue a repercussion listed within the Infraction Repercussion Table. If a desired repercussion strays from the Infraction Repercussion Table then the Election Committee can make a recommendation and only by 3/4 majority, the senate may issue a repercussion that strays from the Infraction Repercussion Table. Consult the Election Procedures and Guidelines Addendum for the possible repercussions. 

\subsection{Result Certification}
The Committee must certify the results of every election of the Senate or 
Cabinet. Should the Committee refuse to certify the results, two-thirds of the 
Senate shall have an opportunity to certify the results by resolution. Should both the Election Committee and the Senate refuse to certify the results, the President shall nominate a member of the student body from each class to certify
the results. 

\section{Public \& Community Relations Committee}

\subsection{Purpose}
The purpose of the Committee shall be to facilitate and enhance active two-way
communication between the SGA and the Student Body, to inform the community of 
any SGA actions and initiatives, to organize events that promote the SGA, and 
to recognize outstanding achievements by all members of the community. The Head of the Public Relations Committee shall be responsible for designating a member of their committee to collaborate with the Oversight Committee for the purpose of creating and/or managing a document containing information that the Senate has deemed worthy of sending to its constituents. This document shall be referred to as the “District Distribution Form.” All Senators shall be able to view and contribute to this document. 
Should the Chair of the Public Relations Committee deem any specific piece of information to be crucial to the knowledge of the Student Body, they may mandate that all Senators communicate that piece of information to their Districts within a certain amount of time that shall be no less than seventy-two (72) hours. This requirement can be overturned by a two-thirds majority vote of the Senate. If the Public Relations Committee Chair mandates that a piece of information be communicated with the Student Body before the next Senate meeting is to be held, they must administer an electronic poll such that the Senate may retain its ability to overturn the requirement.

\section{Campus Life Committee}

\subsection{Purpose}
The purpose of the Committee shall be to ensure the demands of the student 
body are represented in all non-academic facets which directly affect the 
quality of life on campus, to provide the opportunity for student input on any 
major policy and service change, to improve the residence life of the student 
body, to improve the student life of commuter students to ensure the dining 
service on campus is meeting the needs and desires of the student body, to 
provide for student input on the operations of all facilities on campus, and 
to define areas of campus improvement that the student body would see as a 
benefit to student life and to recommend the use of the Capital Improvement 
Fund to address these areas.

\subsection{Student Center Liaison}
A member of the Campus Life Committee shall coordinate with Student Life in 
representing student interests for aspects of a student center delegated to 
the students, and any replacement student center constructed or delegated to 
the students.

\section{The Student Faculty Alliance}

\subsection{Purpose}
To unite selected Stevens’ students, faculty, and academic administrators who have
demonstrated diplomatic ability, uncommon foresight and selflessness. These parties
should share the interests of adding value to the Stevens’ education, improving student-faculty
communication, and introducing a common forum for positive progression both
in and out of the classroom.

\subsection{Incorporation}
The Student Faculty Alliance shall operate as a committee of the Student Government
Association, reporting to and managed by the Vice President of Academic Affairs of the
Student Government Association at Stevens Institute of Technology.

\subsection{Chair}
The elected President of the Student Faculty Alliance shall serve as the chair of the
Student Faculty Alliance and shall be elected by all procedures outlined in Addendum
X: The Operations of the Student Faculty Alliance.

\section{Committee on Academic Rights}

\subsection{Purpose}
The purpose of the Committee on Academic Rights is to assist and aid the current
systems on the campus of Stevens Institute of Technology that promote and uphold the
values and rights of students under the Stevens Honor System.

\subsection{The Honor Committee of the Gear and Triangle Honor Society}
The Committee on Academic Rights will work in tandem with the Honor Committee of
the Gear and Triangle Honor Society to promote the Honor System as a whole on the
campus of Stevens Institute of Technology through events, awareness, and other
methods. 

\subsection{The Stevens Honor Board}
The Committee on Academic Rights will serve to assist the Stevens Honor Board in all
endeavors “to promote honor and integrity throughout the Stevens campus both
academically and socially.”

\subsection{Meetings}
The chairperson of the Committee on Academic Rights will propose a meeting at least
twice per semester with representatives from the Committee on Academic Rights, the
Gear and Triangle Honor Committee, and the Honor Board to discuss progress, establish
goals, and create initiatives to facilitate the Stevens Honor System on campus. 

\section{Academic and Curriculum Advancement Committee}

\subsection{Purpose}
The purpose of the Academic and Curriculum Advancement Committee is to represent
the needs and concerns of the Stevens student, and work with the Faculty Senate to
make certain that University educational endeavors are pursued with the student body
in mind.

\subsection{Establishment}
The Academic and Curriculum Advancement Committee will serve under the supervision
of the Vice President of Academic Affairs of the Student Government Association at
Stevens Institute of Technology. 

\subsection{Chairman and Representative to the Faculty Senate}
The chairman of the Academic and Curriculum Advancement Committee will be
appointed by the Vice President of Academic Affairs. This chairman shall serve as the
Student Government Association representative to the Faculty Senate or shall delegate
one Senator to serve as such in their place.

\section{The Senate Budget Committee}

\subsection{Purpose and Powers}
While the power to transfer money between SGA
accounts and Campus Organization accounts is vested in the Senate, the purpose of the Senate Budget Committee of the SGA shall be to make recommendations to the Senate regarding fund allocations. The Senate Budget Committee shall also have the power to independently approve Reallocation Requests. A Reallocation Request allows a campus organization to use fund
allocations from the SGA on items and services different than the original intent.

\section{Festival Committee}

\subsection{Purpose}
The purpose of the Festival Committee shall be to plan and coordinate a 
campus-wide festival each semester. The Festival Committee shall cooperate 
with and be assisted by EC in planning and coordinating of the campus-wide 
events for the festival.

\subsection{Call for Proposal Submissions}
Requests for proposal submissions for the next semester's festival shall be publicized no later than six (6) weeks before budgets are due for Recognized Student Organizations, with the submission form opening at the same time.

All Submissions that contain all the requested information prior to the deadline set by the Festival Guidance Committee (at the time submissions are opened) shall be viewed by the Festival Guidance Committee.

\subsection{Selection and Funding}
\begin{enumerate}
  \item The Festival Guidance Committee: The Festival Guidance Committee shall be a joint committee of the Entertainment Committee, the Student Government Association, and the Office of Student Life charged with assisting those interested in forming the Festival Committee.
    \begin{enumerate}
      \item Respresentatives from the Entertainment Committee for the Festival Guidance Committee shall be comprised of the Primary Officers of the EC and one Event Officer (as described in their Bylaws).
      \item Representatives from the Student Government Association for the Guidance Committee shall be comprised of the five SGA Cabinet Members and the Speaker of the Senate.
      \item Representatives from the Office of Student Life on the Festival Guidance Committee shall be comprised of the Advisors for the SGA and Entertainment Committee, as well as the Financial Coordinator.
      \item The Festival Guidance Committee shall be chaired by a member of the committee elected at the first meeting while quorum is achieved.
    \end{enumerate}
  \item Funding
    \begin{enumerate}
      \item The Festival Guidance Committee Meeting
        \begin{enumerate}
          \item The Festival Guidance Committee shall meet within one week of the due date of submissions to hear proposals from Submitters and their Committees.
          \item Unless the Festival Guidance Committee, by unanimous consent while quorum is achieved, elects to reject a proposal, all proposals will be provided DuckSync Budgeting access to formally request funding, and further opportunities to meet with the Festival Guidance Committee to strengthen bid proposal.
        \end{enumerate}
      \item Budegts Submitted
        \begin{enumerate}
          \item All proposals that are not rejected by the Guidance Committee must submit a Budget Request, on DuckSyncfor their festival proposal by the deadline for the upcoming semester budgeting period.
        \end{enumerate}
      \item Senate Funding Approval
        \begin{enumerate}
          \item Prior to the approval of RSO Budgets during the budgeting period, all proposals that have submitted budgets shall come before the Senate as a request for funding.
          \item The Senate has the ability to grant funding at the level they desire to as many or as few festival proposals as they deem fit.
          \item After Festival funding is approved, remaining funds shall be disbursed amongst RSO and Organization budget requests as the Senate sees fit.
        \end{enumerate}
    \end{enumerate}
  \item Follow Up
    \begin{enumerate}
      \item Once Funding is approved, the Festival Committee shall be officially formed. The Vice President of Finance and the EC Festivities Officer shall be ex-officio non-voting members of the Festival Committee.
      \item The Festival Committee shall be responsible for meeting all deadlines set forth by the SGA, Entertainment Committee, and the Office of Student Life.
    \end{enumerate}
  \item Removal
    \begin{enumerate}
      \item If a motion for removal/dissolution of the Festival Committee is raised in either the Entertainment Committee or the Senate, and the motion receives a two-thirds vote while quorum is achieved, the motion will carry over to the other governing body, and they will be dissolved by a two-thirds vote while quorum is achieved of the second governing body.
      \item If the Festival Committee is dissolved, the Festivities Officer of the Entertainment Committee shall replace the dissolved festival committee and becomes chair of the Festival Committee. The committee shall be made up of members of the choosing of the Festivities Officer. The new committee can elect to either continue with the approved budgeted festival or submit a Reallocation Request/Additional Funding Request as they deem fit.
    \end{enumerate}
  \item No Proposals Received
    \begin{enumerate}
      \item If no Undergraduate Student Body submissions are submitted, then EC will be in charge of creating a committee, headed by the Festivities Officer, to run the festival. The committee shall be open to the general student body.
      \item A Budget Request must still be made according to the timeline set down in Part 2 (Funding) Section B and C.
    \end{enumerate}
\end{enumerate}

\section{The Student Affairs Committee}

\subsection{Purpose}
The purpose of the Student Affairs Committee will be to create a corps of liaisons to the
various offices at Stevens Institute of Technology. The offices required to liaison to will be
determined by the Chair of the Student Affairs Committee dependent upon relations already
maintained with these offices and their activities. These liaisons will be responsible for any
feedback or change on behalf of their office and will be regarded the “experts” related to
matters which concern their office.

\subsection{Chair and Membership}
The Chair of the Student Affairs Committee will be the Vice President of Operations.
Liaisons will be appointed by the Vice President of Operations, and will have the responsibility of beginning and continuing a relationship between their respective Office and that of the SGA. Liaisons shall become well versed in the operating procedures, goals, principles, and other aspects of the office that they interface with such that they may effectively help with any feedback or other communications necessary.

\section{The Health and Wellness Committee}

\subsection{Purpose}
The purpose of the Health and Wellness Committee shall be to represent the needs and
concerns of the Stevens student body in regards to their mental, physical, and sexual
wellbeing. The committee shall work with the campus health center and administration to
accommodate students’ requests for health services. The committee shall also work
collaboratively with the Office of Counseling and Psychological Services, the wellness
center, Campus Recreation, the Title IX Coordinator, and other organizations to sponsor
and run events on campus that address areas of mental, physical, and sexual health as
well as act as a liaison between these offices and the student body. The committee shall also serve as a channel for students who are interested in facilitating discussions and workshops that serve to educate the student body on mental, physical, and sexual health to do so. 

\chapter{Addendums to These Bylaws} \label{sec:addendums}
An addendum is any supplementary policy or set of policies adopted by the Senate that is to exist separate from the Constitution of the SGA and these bylaws. No addendum may supersede or be inconsistent with the Constitution of the SGA or these bylaws. 
Addendums shall be adopted after the procedure for passing a bill is followed. 
Amendments to an addendum shall occur after the procedure for passing a bill is followed. An addendum shall be removed from law after the procedure for passing a bill is followed.

\chapter{Amendments to These Bylaws} \label{sec:amendments}
Amendments to these bylaws shall occur after the procedure for passing a bill 
is followed. 

\chapter{Implementation}
Upon ratification of these bylaws, all standing committees and sub-committees 
will be organized as described in these bylaws. These bylaws shall take effect 
immediately after ratification and all further items of business at that 
Senate meeting shall follow these bylaws. 

\chapter{Suspension of These Bylaws} \label{sec:suspension}
No bylaw may be suspended, except by a majority vote of the Senate and 
approval of the President. \ref{sec:addendums}, \ref{sec:amendments}, \ref{sec:suspension}, 
\ref{sec:senate_powers}, and \ref{sec:rso_powers} of these bylaws may never be 
suspended. Any suspension of the bylaws shall only last until the end of the 
Senate meeting in which the bylaws are suspended. 

\chapter{Miscellaneous}

\section{Written Notice}
Only a physical paper letter or a message sent from a Stevens-issued email account shall satisfy any provision
requiring written notice.

\section{Capital Improvements and Infrastructure}
At least three percent of the total amount of student activity fees collected 
by the SGA shall be saved for capital expenditures benefiting the campus. 
These expenditures need not occur during the semester the money is collected, 
but the money set aside shall not be eligible for allocation for any purposes 
other than capital improvements during later semesters. 

However, if the Capital Improvements Fund has accumulated more than fifty 
percent of the Student Activities Fund from the most recent fall semester, 
only two percent of the collected student activity fees shall be diverted to 
the Capital Improvements Fund, unless the Senate passes a standing motion to 
the contrary. 

\section{Headings}
The heading under which text falls, such as section title or article name, 
shall not bound the effect of that content. 

\section{Senior Year Activity Allotment}
Five percent of the student activity fees of each class year, as defined by 
the Constitution of the SGA, shall be set aside into senior year activities 
accounts bound to the graduating class from which the fees were collected. 
This account shall be unavailable for expenditure until the class year to 
which it is bound enters its senior year. The Senior Year Activity Committee, 
by a vote of two-thirds of committee quorum, shall have the power to spend 
from the account once the account's bound year enters its senior year. 

\section{Advertising}
The SGA shall be responsible for informing the Student Body of all campus 
initiatives and for assisting all RSOs in publicizing their events. 

The SGA shall fund flyer requests for all RSOs. Any additional advertising 
requests, as well as any requests by non-RSO Campus Organizations, must be 
approved by the Senate. 

\chapter{Severability}
If a court of competent jurisdiction deems any provision of these bylaws 
unenforceable, that provision will be enforced to the maximum extent 
permissible, and remaining provisions will remain in full force and effect. 

\chapter{Powers Reserved to the Senate} \label{sec:senate_powers}
The Senate shall have the power to carry forth acts when explicitly permitted 
by these bylaws. Any further actions shall require amendment to these bylaws 
to permit these acts. 

\chapter{Powers Reserved to Recognized Student Organizations} \label{sec:rso_powers}
Any powers not explicitly reserved in these bylaws for the Senate and not 
explicitly banned in these bylaws shall be reserved to the Recognized Student 
Organizations.

\end{document}

%%% Local Variables:
%%% mode: latex
%%% TeX-master: t
%%% End:
